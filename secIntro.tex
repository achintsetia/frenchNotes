\section{Introduction}

\subsection{Alphabets}
\noindent Pronunciation in english is provided in the square brackets.
\vspace{10pt}

\begin{tabular}{l l l l l l}
a [ah] & b [bay] & c [say] & d [day] & e [uh] & f [ef] 		\\
g [jzay] & h [ash] & i [ee] & j [jzee] & k [kah] & l [el] 	\\
m [em] & n [en] & o [oh] & p [pay] & q [kew] & r [er] 		\\
s [es] & t [tay] & u [ew] & v [vay] & w [doobluh vay] & x [eeks] \\
y [eegrek] & z [zed] \\
\end{tabular}

\subsection{Counting}

\begin{tabular}{| l | l | l | l | l | l |} \hline
1  & un 	& $a^n$		& 11 & onze 	& $oh^nz$ 	\\ \hline
2  & deux	& $duh$		& 12 & douze 	& $dooz$	\\ \hline
3  & trois	& $trwa$	& 13 & treize	& $trehz$ 	\\ \hline
4  & quatre	& $katruh$	& 14 & quatorze & $katroz$	\\ \hline
5  & cinq	& $sa^nk$	& 15 & quinze	& $ka^nz$ 	\\ \hline
6  & six	& $sees$	& 16 & seize	& $sehz$ 	\\ \hline
7  & sept	& $set$		& 17 & dix-sept & $deeset$	\\ \hline
8  & huit	& $weet$	& 18 & dix-huit & $deezweet$\\ \hline
9  & neuf	& $nuhf$	& 19 & dix-neuf & $deenuhf$	\\ \hline
10 & dix	& $dees$	& 20 & vingt	& $va^n$	\\ \hline
\end{tabular}

\vspace{.5in}

\begin{tabular}{| l | l | l | l | l | l |}
\hline
20  & vingt         & 30  & trente          & 40  & quarante\\
50  & cinquante     & 60  & soixante        & 70  & soixante-dix\\
80  & quatre-vingt  & 90  & quatre-vingt-dix& 100 & cent\\
\hline
\end{tabular}

\subsection{Ordinal Numbers}
\begin{tabular}{| l | l | l | l |}
\hline
$1^{st}$  & premier/premi\`ere  & $2^{nd}$   & deuxi\`eme   \\
$3^{rd}$  & troisi\`eme         & $4^{th}$   & quatri\`eme  \\
$5^{th}$  & cinqui\`eme         & $6^{th}$   & sixi\`eme    \\
$7^{th}$  & septi\`eme          & $8^{th}$   & huiti\`eme   \\
$9^{th}$  & neuvi\`eme          & $10^{th}$  & dixi\`eme    \\
\hline
\end{tabular}

\subsection{Weekdays (Les jours de la semaine)}

\begin{tabular}{| l | l | l |}
\hline
Monday    & lundi     & (luhn-DEE)      \\
Tuesday   & mardi     & (mahr-DEE)      \\
Wednesday & mercredi  & (mehr-kruh-DEE) \\
Thursday  & jeudi     & (juh-DEE)       \\
Friday    & vendredi  & (vahn-druh-DEE) \\
Saturday  & samedi    & (sahm-DEE)      \\
Sunday    & dimanche  & (dee-MAHNSH)    \\
\hline
\hline
day before yesterday & avant-hier & (aw-van-yehr) \\
yesterday & hier        & (YEHR)          \\
today     & aujourd'hui & (aw-zhoor-DWEE) \\
tomorrow  & demain      & (duh-MANG)    \\
day after tomorrow & apr\`{e}s-demain & (apre-duh-MANG) \\
this week & cette semaine & (set SMEN)  \\
last week & la semaine dernière   & (lah SMEN dehr-NYEHR)\\
next week & la semaine prochaine  & (lah SMEN proh-SHEN)  \\
\hline
\end{tabular}

\subsection{Months of Year (Le Mois De L'ann\'ee)}

\begin{tabular}{| l | l | l |}
\hline
January     & Janvier       & (zhahng-VYAY) \\
February    & F\'evrier     & (fay-VRYAY)   \\
March       & Mars          & (mahrs)       \\
April       & Avril         & (ah-VREEL)    \\
May         & Mai           & (meh)         \\
June        & Juin          & (zhwang)      \\
July        & Juillet       & (zhwee-YAY)   \\
August      & Ao\^ut        & (oot)         \\
September   & Septembre     & (set-TAHMBR)  \\
October     & Octobre       & (ock-TOHBR)   \\
November    & Novembre      & (noh-VAHMBR)  \\
December    & D\'ecembre    & (day-SAHMBR)  \\
\hline
\end{tabular}

\subsection{To Specify Date and Time (Pr\'eciser la date et l'heure)}

\begin{itemize}
\item When we use numbers for date, we write the article \emph{le} before the number.\\
C\'elia est n\'ee le 3 f\'evrier 1970.
\item When we specify month or year, we use preposition \emph{en}.\\
Je suis n\'e en Mai.\\
Je suis n\'e en 2010.
\item To specify a duration (la dur\'ee), we use preposition \emph{de} and \emph{\`a}.\\
de (from) 3h \`a (to) 6h\\
du 13 Septembre au 21 D\'ecembre.
\item To specify time we use the following:\\
\begin{tabular}{l l l}
12pm (noon) & midi\\
12am (mid-night) & minuit\\
13h & 1h d'apr\`es-midi\\
14h & 2h d'apr\`es-midi\\
18h & 6h du soir  \\
20h10 & 8h10 du soir\\
20h15 & 8h15 du soir & 20h et quart \\
20h30 & 8h30 du soir & 20h et demie \\
20h45 & 8h45 du soir & 9h moins le quart du soir \\
20h55 & 8h55 du soir & 9h moins cinq du soir \\
22h & 10h du soir \\
\end{tabular}\\
with \emph{moins} we use the article \emph{le} with \emph{quart} and \emph{demie}.
\item We use \emph{\`a l'heure} for saying on time,\\
\emph{en avance} for saying early,\\
and \emph{en retard} for saying late.
\end{itemize}

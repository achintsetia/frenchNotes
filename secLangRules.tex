\section{French Language Rules}

\begin{itemize}

\item{Every noun in French has a gender.}

\item{After \emph{beaucoup}, we always add \emph{de}. e.g.\ Il y a beaucoup de
robes pour toi.}

\item{All languages are considered masculine. e.g.\ le fran\c{c}aise, le polonais
etc.}

\item{Unlike english, we never put an article with person's:
1) occupation, Il est serveur, Je suis etudiant, Je suis chanteure,
2) religion, elle est musulmane, and
3) hundred (cent) and thousand (mille), cent cinq (105).}

\item{\emph{gn} and \emph{ll} are always pronounced as \emph{y}. e.g.\ Espagnol and fille(girl).}

\item{Ville (city) is always feminine. e.g.\ La ville de Paris.}

\item{Only one can exist in a sentence. e.g.\ C'est le stylo de Marie, Ce sont les livres de Pierre.}

\item{Most nouns add $-s$ to form the plural (une chaise - des chaises).
The $-s$ is not pronounced.}

\item{Nouns ending in $s, x,$ or $z$ do not change in the plural:
\emph{un prix} (a price) - \emph{des prix} (some prices).}

\end{itemize}

\section{French Language Rules}

\begin{itemize}

\item{Every noun in French has a gender.}

\item{After \emph{beaucoup}, we always add \emph{de}. e.g.\ Il y a beaucoup de
robes pour toi.}

\item{All languages are considered masculine. e.g.\ le fran\c{c}aise, le polonais
etc.}

\item{Unlike english, we never put an article with person's:
1) occupation, Il est serveur, Je suis \'etudiant, Je suis chanteure,
2) religion, elle est musulmane, and
3) hundred (cent) and thousand (mille), cent cinq (105).}

\item{\emph{gn} and \emph{ll} are always pronounced as \emph{y}. e.g.\ Espagnol and fille(girl).}

\item{Ville (city) is always feminine. e.g.\ La ville de Paris.}

\item{Only one article can exist in a sentence. e.g.\ C'est le stylo de Marie, Ce sont les livres de Pierre.}

\item{Most nouns add $-s$ to form the plural (une chaise - des chaises).
The $-s$ is not pronounced.}

\item{Nouns ending in $s, x,$ or $z$ do not change in the plural:
\emph{un prix} (a price) - \emph{des prix} (some prices).}

\item{All adjectives that end with \emph{``nal''}, their plural is formed by removing
\emph{``l''} and adding \emph{``ux''}\\
e.g. International - Internationaux, Journal - Journaux}

\item{There are only few adjectives that we put before the noun.\\
grand (big, tall), petit (small)	 \\
bon (good), mauvais (bad) \\
meilleur (better), jeune (young)\\
vieux (old), autre (other) \\
beau (beautiful), faux (false, fake) \\
gros (large, fat), haut (high, tall) \\
joli (pretty), même (same) \\
nouveau (new)}

\item{We use \emph{de} instead of \emph{des} if:\\
- There is an adjective before the noun\\
- The noun is plural\\
Un beau tableau - De beaux tableaux,\\
Un bon journal - De bons journaux.}

\item{We use the verb \emph{jouer (to play)} only with sports, music instruments, and games.
For all other activities we use \emph{Faire (to do)}.}

\end{itemize}
